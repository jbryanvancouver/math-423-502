\documentclass[12pt]{article}

\usepackage{amsfonts, amsmath, amsthm}
\usepackage{tikz-cd}
\usepackage{amsmath,amscd}
\usepackage{verbatim}
\addtolength{\evensidemargin}{-01in}
\addtolength{\oddsidemargin}{-01in}
\addtolength{\textwidth}{2in}
\addtolength{\textheight}{2in}
\addtolength{\topmargin}{-.5in}


%\begingroup
%\catcode`\&=13
%\gdef\pampmatrix{%
%  \begingroup
%  \let&=\amsamp
%  \begin{pmatrix}%
%}
%\gdef\endpampmatrix{\end{pmatrix}\endgroup}
%\endgroup


\newcommand{\ZZ}{\mathbb{Z}}
\newcommand{\CC}{\mathbb{C}}
\begin{document}


\begin{center}{\Large
\vspace{2cm}

\bf  Homological Algebra Homework problem set 0\quad   Math 423\quad  \\
}
\vspace{.5in}

\end{center}

In the following problems, elements of direct sums will be considered
as row vectors and the maps will be expressed as matrices, acting on
row vectors by multiplication on the right. 

In your answers, express all vector spaces and Abelian groups in
standard form. Namely, all (finite dimensional) complex vector spaces
are of the form $\CC^{d}$ for some $d\geq 0$, and all (finitely generated)
Abelian groups are products of cyclic groups, i.e. of the form
$\ZZ^{d}\times (\ZZ/p_{1})^{a_{i}}\times \dotsb \times
(\ZZ/p_{k})^{a_{k}} $ for some $d\geq 0$, primes $p_{1},\dotsc
,p_{k}$, and $a_{1},\dotsc ,a_{k}> 0$.  

\begin{enumerate}
\item Compute the homology groups of the following complex of
$\CC$-modules (a.k.a. complex vector spaces). The first non-trivial
one is in degree 3, the last is in degree 0:

\[
\begin{tikzcd}[column sep=large,ampersand replacement=\&]
0 \arrow{r} \&
\CC^{2} \arrow{r}{
 \left( \begin{smallmatrix} 9 & 9 \\ 6 & 6 \end{smallmatrix}  \right) } \& 
\CC^{2} \arrow{r}{ \left( \begin{smallmatrix} -5 & 10 \\ 5 & -10 \end{smallmatrix}  \right) } \& 
\CC^{2} \arrow{r}{  \left( \begin{smallmatrix} 10 & 6 \\ 5 & 3 \end{smallmatrix} \right) } \&
\CC^{2} \arrow{r} \& 0 
\end{tikzcd}
\]
\item Compute the homology groups of the following complex of
$\ZZ $-modules (a.k.a. Abelian groups). The first non-trivial
one is in degree 3, the last is in degree 0:

\[
\begin{tikzcd}[column sep=large,ampersand replacement=\&]
0 \arrow{r} \&
\ZZ^{2} \arrow{r}{
 \left( \begin{smallmatrix} 9 & 9 \\ 6 & 6 \end{smallmatrix}  \right) } \& 
\ZZ^{2} \arrow{r}{ \left( \begin{smallmatrix} -5 & 10 \\ 5 & -10 \end{smallmatrix}  \right) } \& 
\ZZ^{2} \arrow{r}{  \left( \begin{smallmatrix} 10 & 6 \\ 5 & 3 \end{smallmatrix} \right) } \&
\ZZ^{2} \arrow{r} \& 0 
\end{tikzcd}
\]


\item Let $R$ be the ring $\CC [x,y]$. Compute
the homology groups of the following complex of $R$ modules
(the first non-trivial module is in degree 2, the last non-trivial
module is in degree 0):
\[
0\to R \xrightarrow{\,\, \alpha  \,\,}  R\oplus R \xrightarrow{\,\, \beta  \,\,}
R   \to 0
\]
where $\alpha : f(x,y)\mapsto (x\cdot f(x,y),y\cdot f(x,y))$ and
$\beta : (g(x,y),h(x,y)) \mapsto y\cdot g(x,y)- x\cdot h(x,y)$.  
\item
Consider map of complexes given below. Compute the homology groups of
each complex and then compute the induced map from the homology groups
of the first complex to the homology of the second.

\[
\begin{tikzcd}[column sep=large,ampersand replacement=\&]
0 \arrow{r}\arrow{d} \&
\ZZ^{2} \arrow{d}{\left(\begin{smallmatrix} 6\\ 4 \end{smallmatrix} \right)} \arrow{r}{
 \left( \begin{smallmatrix} 9 & 9 \\ 6 & 6 \end{smallmatrix}  \right) } \& 
\ZZ^{2}  \arrow{d}{\left(\begin{smallmatrix} 5\\ -1 \end{smallmatrix} \right)} \arrow{r}{ \left( \begin{smallmatrix} -5 & 10 \\ 5 & -10 \end{smallmatrix}  \right) } \& 
\ZZ^{2}  \arrow{d}{\left(\begin{smallmatrix} 2\\ 1 \end{smallmatrix}
\right)} \arrow{r}{  \left( \begin{smallmatrix} 10 & 6 \\ 5 & 3 \end{smallmatrix} \right) } \&
\ZZ^{2}  \arrow{d}{\left(\begin{smallmatrix} 2\\ 2 \end{smallmatrix} \right)} \arrow{r} \& 0 \arrow{d} \\
0 \arrow{r} \&
\ZZ \arrow{r}{ \cdot 6 } \& 
\ZZ \arrow{r}{\cdot 0 } \& 
\ZZ \arrow{r}{ \cdot 16 } \&
\ZZ \arrow{r} \& 0 
\end{tikzcd}
\]



\end{enumerate}





          
\end{document}
